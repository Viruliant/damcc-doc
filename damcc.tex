``The idea that you can do a programming language in 100 000 lines of
code or you might be able to do it in a page and there are advantages to
the page for sure'' (Kay 2009)

\hypertarget{definitional-architecture-metaprogramming}{%
\section{Definitional Architecture
Metaprogramming}\label{definitional-architecture-metaprogramming}}

\hypertarget{definitional-interpreters-di}{%
\paragraph{Definitional-Interpreters
(DI)--}\label{definitional-interpreters-di}}

convey the ideas and semantics of a language with brevity and clarity;
the advantage with defining a language with this method over a
specification document is that the result of the \textbf{DI} is
executable, so there is no question of the meaning of an expression in a
context.

\hypertarget{architecture}{%
\paragraph{Architecture--}\label{architecture}}

Different cpus need different executables if they use different
instruction-sets, and even if they do use the same
\textbf{ISA}(instruction-set-architecture), they may have new Extensions
that may extend an \textbf{ISA}, to let older programs run on newer
cpus.

\hypertarget{metaprogramming}{%
\paragraph{Metaprogramming--}\label{metaprogramming}}

technique of programs manipulating other programs as data.

\hypertarget{architecture-data-structure-c}{%
\paragraph{\texorpdfstring{Architecture Data-Structure ``\texttt{C}''
--}{Architecture Data-Structure ``C'' --}}\label{architecture-data-structure-c}}

For compiling/interpreting any language; \texttt{C} is an excellent
target architecture to use as a stepping stone to actual compiled
assembly code. ``\texttt{damcc}'' is not just a \textbf{DI} for
\texttt{C} it is also to serve as documentation for the different
\textbf{ISA}s. Users should be able to understand an architecture
quickly and easily by reading the code for the supported
\textbf{ISA}s.(as well as elf files and static/dynamic linking)
\emph{``It is better to have 100 functions operate on one data structure
than 10 functions on 10 data structures.''} --Alan Perlis

\hypertarget{metaprogramming-with-ometa}{%
\subsection{\texorpdfstring{Metaprogramming with
\textbf{OMeta}}{Metaprogramming with OMeta}}\label{metaprogramming-with-ometa}}

See (Warth 2009) for what OMeta Is, this section is about how it works.

(Piumarta 2007) -- \emph{``The system that tries to find coverings of
those trees pieces of structure within the trees, is itself simple and
you can write 2 functions that do it.}

\begin{verbatim}
reduce(tree, startSymbol) =
    foreach rule in startSymbol.startSets
        if match(tree, rule.pattern)
            rule.action()
            return startSymbol
    return false
match(tree, pattern) =
    if (pattern.isSymbol()) return reduce(tree, pattern)
    if (tree.first != pattern.first) return false
    foreach treeElement, patternElement in tree.tail, pattern.tail
        unless match(treeElement, patternElement)
            return false
    return true
\end{verbatim}

\emph{\textbf{Reduce} a particular piece of tree to a quantity: a
register in the machine, a memory location, a literal in the machine, or
a \texttt{void} just a statement that returns no value.}

\emph{\textbf{Match} a particular piece of tree against the pattern in
the list of rules. These things are recursive so that structuring the
trees can have sub structure etc. and what falls out at the end is a yes
or no answer.}

\emph{It's the inverse of traditional parsing if you like where a normal
parser takes unstructured free text and creates a structured
representation. What this bottom-up system does is it takes a structured
representation and tries to find ``an optimal unstructured equivalent''
which in our case are machine instructions, for the for the target
machine.}

\emph{One of the interesting things that we've now done is we've
described code generation as a tree rewriting process. . . . Various
phases of the implementation of a programming system can be described in
terms of tree like structures and rewriting-rules applied to them. We
can hope in the end to create a single small easily understandable tree
rewriting system in which the whole of the language implementation
problem is described. . . . Then everything else the the transformation
rules for taking structures applying all the functional rules to them to
create executable things within fed in and from that point on the whole
system takes off exponentially because of the simplicity the
expressiveness that is built into those 2 mutually supportive models
within it.''}

\hypertarget{elf-format-relocatable-object-.o-files}{%
\subsection{\texorpdfstring{(ELF Format) Relocatable Object
``\texttt{.o}'' Files
--}{(ELF Format) Relocatable Object ``.o'' Files --}}\label{elf-format-relocatable-object-.o-files}}

Relocatable object files are files that are in the ``elf'' executable
and linkable format. To Dynamically link to each foreign function call
requires parsing header files to see which header provides each function
that is used in the current file, to properly call the function and
throw an error if a function isn't available. ``\texttt{.o}'' files are
binary data assembly with symbols packed in, based on a specification.
This is the format for linux executables and libraries, These files are
the raw inputs to ``linkers''. Translating from ``\texttt{C}'' to
``\texttt{.o}'' linkable binary files is the goal here and the
definition of compiling.

\hypertarget{stage-1-explicitly-enumerate-the-nuances-of-c-by-removing-them}{%
\paragraph{\texorpdfstring{Stage 1 ``explicitly enumerate the nuances of
\texttt{C}'' by ``removing
them''}{Stage 1 ``explicitly enumerate the nuances of C'' by ``removing them''}}\label{stage-1-explicitly-enumerate-the-nuances-of-c-by-removing-them}}

Define what the \texttt{C} language is using ``Ometa'' to convert from
preprocessed \texttt{C} to \texttt{C} or our small subset of it as an
Intermediate Representation(IR) The goal of using this subset of
\texttt{C} is that it is more easily translatable to machine code.

\begin{itemize}
\tightlist
\item
  Nested if statements translated into guard clause non-nested if
  statements.
\item
  files referenced by \texttt{\#include} will have their functions
  scanned and added to a table of functions listing the file and the
  functions and types so that when a external function is called the
  appropriate ``\texttt{symbol()}'' call of the elf file can be made. If
  there isn't a function available that accepts the given type abort
  with error.
\item
  \textbf{Memory--} all of variables \textbf{AND} pointers Each function
  or block \texttt{\{\}} is to have all memory explicitly
  allocated/deallocated with appropriate calls to
  \texttt{malloc()\ realloc()\ calloc()\ aligned\_alloc()\ free()}
\item
  \textbf{goto --} The meaning of all loops translated into gotos. If an
  active block is exited using a goto statement, explicitly
  \texttt{free()} memory for any local variables before control is
  transferred from that block.
  --\href{https://www.ibm.com/docs/en/zos/2.1.0?topic=statements-goto-\%20statement}{ibm
  goto}
\item
  even though we can replace function calls with gotos, they need to be
  left in to represent where we will leave symbol references for the
  ``\texttt{.o}'' files (even executables are formatted this way the
  \texttt{main()} function is called).
\item
  all paramaters to functions that are not passed by reference are to be
  converted to paramaters that are passed by reference, and that
  variable is never to be referenced, except when it is copied to
  another location at the begining of the function and that copy is to
  be used in its place in all of it's occurences.
\item
  \texttt{sizeof()} tells the number of bytes that type or array
  occupies, and \texttt{sizeof()} is known at compile time, and most of
  the time this number is fed into arithmetic functions that can also be
  calculated at compile time; which can allow optimizations here.
\item
  arithmetic-assignment-operators (\texttt{+=}, \texttt{-=},
  \texttt{*=}, \texttt{/=}, \texttt{\%=}) need to be converted to a
  combination of the arithmetic operation of the correct type and the
  assignment.
\item
  Each arithmetic-operator-function (\texttt{+}, \texttt{-}, \texttt{*},
  \texttt{/}, \texttt{\%}), and each unary-operator-function
  (\texttt{++}, \texttt{-\/-}), and each bitwise-operator-function
  (\texttt{\&}, \texttt{\textbar{}}, \texttt{\textless{}\textless{}},
  \texttt{\textgreater{}\textgreater{}}, \texttt{\textasciitilde{}},
  \texttt{\^{}}) and each boolean-logical-operator (\texttt{\&\&},
  \texttt{\textbar{}\textbar{}}, \texttt{!}) as well as relational
  operators (\texttt{\textless{}}, \texttt{\textless{}=},
  \texttt{\textgreater{}}, \texttt{\textgreater{}=}, \texttt{==},
  \texttt{!=}) must be converted to a function call named after the type
  of the inputs.
\item
  translate the \texttt{-\textgreater{}} operator into pointer
  arithmetic
\item
  get the \texttt{...} primitive in
  \texttt{foo(sometype\_t\ var1,\ ...)} to work with \texttt{stdarg.h}
  somehow ???
\item
  type and pointer arithmetic for the \texttt{-\textgreater{}} primitive
  to not need to be parsed.
\item
  \textbf{Strongly Typed--} force \texttt{anytype\_t} to be
  \texttt{sometype\_t} with:\\
  \texttt{*(sometype\_t*)\&(anytype\_t-code)}, should enlighten users
  why the following 2 prints aren't equal.
\end{itemize}

\emph{``Start at the variable name (or innermost construct if no
identifier is present. Look right \textbf{without jumping over a right
parenthesis}; say what you see. Look left again \textbf{without jumping
over a parenthesis}; say what you see. Jump out a level of parentheses
if any. Look right; say what you see. Look left; say what you see.
Continue in this manner until you say the variable type or return
type.''}
--\href{https://parrt.cs.usfca.edu/doc/how-to-read-C-declarations.html}{Terrence
Parr}

\begin{verbatim}
float64_t rydb = 123456789543211;// hex code for rydb with typedef
printf("hex value of \"rydb\" is %"PRIX64"\n", rydb);
printf("hex value of \"rydb\" is %"PRIX64"\n", *(uint64_t*)&rydb);
\end{verbatim}

\begin{itemize}
\tightlist
\item
  no more Arrays, Types, Struct, Union, Enum, or any other data-types
  not manipulatable with assembly; just pointer arithmetic.
\end{itemize}

\hypertarget{pointers}{%
\paragraph{pointers --}\label{pointers}}

\textbf{``address of'' operator: ``\texttt{\&}'' --} returns address in
memory of a variable

\textbf{``dereference'' operator: ``\texttt{*}'' --} Grab data of type
of the argument at the given memory location

There is no limit to the depth of the pointer to pointer to pointer to
pointer type.

Do typecasting and pointer arithmetic \textbf{WITH THE REQUESTED
PADDING} for: structs, arrays, enum, and union etc. and padding:
according to \texttt{\#pragma\ pack(x)} fed to stage1. Rewrite struct
with \texttt{\#pragma\ \ pack(1)} and redeclare \texttt{struct} members
in order decreasing size, and do padding according to \texttt{x}.
sizeof() the type of the data being stored in the array times length of
the array. When a data type is naturally aligned, the CPU fetches it in
minimum read cycles.

\begin{itemize}
\tightlist
\item
  the \textbf{ternary operator} (\texttt{?:}) is to be converted to if
  statements, and function calls to the appropriate boolean function.
\item
  assignment operator \texttt{=}
\end{itemize}

\hypertarget{inline-functions}{%
\paragraph{Inline functions --}\label{inline-functions}}

The instructions of a function have there own place in memory and a jump
to that section is performed followed by a jump back to wherever the
function was called from. Short function calls that can be ``inlined''
can take less cpu clocks to run than jumping to some other section of
memory. \textbf{Inline functions} mean the whole code of the function is
written verbatim whereever it is called without any jump instructions.
In certain situations some compilers and may decide not to honor the
``\texttt{inline}'' primitive.

\begin{verbatim}
// Static Inline function for each possible compare or jump
// for the current architecture in assembly
static inline int foo(){
    return 2;
}
\end{verbatim}

At each stage of compilation/translation \texttt{damcc} data will remain
in C until there is nothing left but assembly.

\begin{center}\rule{0.5\linewidth}{0.5pt}\end{center}

\begin{verbatim}
#                                                               #
#                                                               #
#                                                               #
#                                                               #
#                                                               #
#                                                               #
#                                                               #
#                                                               #
#                                                               #
#                                                               #
#                                                               #
#                                                               #
#                                                               #
\end{verbatim}

\hypertarget{metaprogramming-example-translator}{%
\paragraph{Metaprogramming Example
translator}\label{metaprogramming-example-translator}}

Control flow operators? maybe all we need is ``\texttt{goto}'' read this
code:

\begin{verbatim}
while (condition) statement
//the above while-loop is equivalent to:
begin:
if !(condition) goto end;
    statements;
    goto begin;
end:

do [loop body statement] while ([condition]);
//the above do-while-loop is equivalent to:
begin:
loop body statement;
if (condition)
    goto begin;

for (variable-declaration;
     condition;
     variable-update)
     statement
//the above for-loop is equivalent to:
variable-declaration;
begin:
    if (!condition)
        goto end;
    statement;
    variable-update;
    goto begin;
end:
    // end loop 
#                          Switch                                #
switch (expression) {
    case const-expr1: statement1
    case const-expr2: statement2
    case const-expr3: statement3
    default: statements
}
//the above switch is equivalent to:
type value = expression
if (value == const-expr1)
    statement1
else if (value == const-expr2)
    statement2
else if (value == const-expr3)
    statement3
else
    statement0

//Which is also equivalent to:
type value = expression
if !(value == const-expr1) goto alternate2;
    statement1; goto end;
alternate2:
if !(value == const-expr2) goto alternate3;
    statement2; goto end;
alternate3:
if !(value == const-expr3) goto alternate0;
    statement3; goto end;
alternate0:
    statement0
end:
#                                                               #
#                                                               #
#                                                               #
#                                                               #
#                                                               #
#                                                               #
if (n > 0) { //nested if
    if (a > b)
        z = a;
}
else
    z = b;

//guard clause version of nested if
if !(n > 0) goto alternate;
if !(a > b) goto alternate;
    z = a;
    goto end;
alternate:
    z = b;
end:
\end{verbatim}

\begin{center}\rule{0.5\linewidth}{0.5pt}\end{center}

\hypertarget{clean-and-convoluted}{%
\paragraph{\texorpdfstring{\emph{``Clean''} and
convoluted--}{``Clean'' and convoluted--}}\label{clean-and-convoluted}}

after Stage1 is complete there will be no need to:

\begin{itemize}
\tightlist
\item
  parse the \texttt{sizeof()} command
\item
  parse types of variables
\item
  parse creation or destruction of any variables -- the \texttt{C} code
  will be requesting memory for it's own variables and pointers the
  appropriate function calls (\texttt{malloc()} etc.).
\item
  create or manage Arrays, Types, Struct, Union, Enum
\item
  parse control flow statements switches loops etc.
\item
  destroy any variables -- the c code should be explicitly calling
  \texttt{free()} on any variables that are to cease existance after
  they fall out of scope including global variables at the end of the
  \texttt{main()} scope.
\end{itemize}

\emph{``I came to realize that if I'm not writing a program, I shouldn't
use a programming language. People confuse programming with coding,
coding is to programming what typing is to writing. It's something that
involves mental effort, what you're thinking about, what you're going to
say, the words have some importance; but in some sense that even they
are secondary to the ideas. In the same way programs are built on ideas,
they have to do something and what they're supposed to do, is like what
writing is supposed to convey. If people are trying to learn programming
by being taught to code, well they're being taught writing by being
taught how to type, and that doesn't make much sense.''} --Leslie
Lamport https://youtu.be/rkZzg7Vowao

\begin{center}\rule{0.5\linewidth}{0.5pt}\end{center}

\hypertarget{c-compiler-damcc-as-the-di-of-the-c-language}{%
\paragraph{\texorpdfstring{``\texttt{C}'' Compiler (\texttt{damcc}) as
the DI of the ``\texttt{C}'' language
--}{``C'' Compiler (damcc) as the DI of the ``C'' language --}}\label{c-compiler-damcc-as-the-di-of-the-c-language}}

Rarely is a language as well put together as \texttt{C}, Mixing
\texttt{C} and assembly can often be as simple as not translating the
assembly parts of the \texttt{C} code. A lot of compromises were made to
do this; \texttt{C} is not so much a language so much as it is basically
assembly and just as dangerous. \texttt{C} is a notoriously ``Dumb''
language, in that it will do exactly what \textbf{you} tell it to do;
Including the set of all known computer \emph{bugs}: memory leaks, stack
overflows, disk destruction, or worse.

\hypertarget{stage-2-register-allocation}{%
\paragraph{Stage 2 ``Register allocation''
--}\label{stage-2-register-allocation}}

determining which values should be placed into which memory hierarchy
and at what time during the execution of the program. The different
storage devices on modern computers: hard disk, ram, cpu cache, \& cpu
registers all have different latencies, and \texttt{C} programs are
written as if there are only 2 kinds of memory: main memory and disk.
The Compiler is to uphold this illusion by adding logic into the program
to move data between the different memory hierarchies: memory,
cpu-cache, and cpu-registers. The compiler is not responsible for the
hard disk since it is accessed by function calls to the system or
kernel.

\emph{``Register allocation phase of the compiler stands between the
optimization phase and the final code assembly and emission phase. When
the intermediate or internal language (IL) enters register allocation,
it is written assuming a hypothetical target machine having an unlimited
number of high-speed general-purpose CPU registers. It is the
responsibility of the optimization phase to eliminate references to
storage by keeping data in these registers, as much as possible.''}
--(Chaitin 1982)

(Click 2020) -- \emph{``\textbf{Outline how The architecture description
files, in how in general made the compiler suitable for many platforms
all at once:} . . . There is an architecture file that is used to
describe a CPU architecture including the set of the kind of
instructions and a mapping from the C2 ideal nodes to machine
equivalents and the registers that are allowed and the encoding for the
machine code encodings for them is sort of the common set of things
there's a bunch of the things that are in there that are all these
fine-grained details but the big pieces are: a mapping from idealized
nodes which is what the IR mostly runs on to code gen to hardware
instructions and the registers that are allowed both on each individual
hardware instruction input and overall what are those set of arc
registers are allowed in the system and there is a tool which reads this
file which is enough funny in its own little format and spits out C code
that is in compiled in with the rest of the system.}

\emph{Builds what's called a burrs pattern matching which does a optimal
for some definition of optimal mapping of Hardware instructions to ideal
nodes by doing like an overlapping tree like like usually a hardware
node covers two to five ideal nodes and so there's some tiling operation
that covers them all including if it eight this region then those edges
in between don't have to register allocated but all the edges on the
outside do and then what all the registers that are involved and the
registers turn into basically a bit set for every input and output of
allowed registers and the register allocator Grox that bit set and uses
it to pick the right registers so that gets into registrar in pretty
heavily right away because that's how you describe inputs to the
register allocator''}

there are similar files in LLVM that describe CPU architectures.

Intermediate Representation (IR)

LLVM target descriptions are typically written in LLVM's own
assembly-like language, also known as the LLVM \textbf{IR}. These files,
called LLVM target descriptions, specify the \textbf{ISA} of the target
machine, register sets, code generation options, and constraints on the
code generation. The LLVM target descriptions are used by the LLVM code
generator to produce machine code for a specific target architecture.

In the LLVM source code, the target descriptions are typically located
in the ``lib/Target'' directory. The ``.td'' extension target
description files are usually found in subdirectories, for example
``lib/Target/X86/X86.td''.

\begin{center}\rule{0.5\linewidth}{0.5pt}\end{center}

\hypertarget{executable-linkers-are-basically-just-home-theater-setups-}{%
\paragraph{Executable Linkers are basically just home theater
setups-}\label{executable-linkers-are-basically-just-home-theater-setups-}}

Executable Linkers are basically just home theater setups CS361 Chris
Kanich
https://youtu.be/eQ0KOT\_J8Sk?list=PLhy9gU5W1fvUND\_5mdpbNVHC1WCIaABbP

in certain situations with low memory budgets other techniques might be
used --https://github.com/avrdudes/avr-libc

\texttt{C99} doesn't support function overloading . . . but we could
perhaps add that as a command arg to stage1 by renaming overloaded
functions to have the names of parameter types in part of the function
name.

\begin{center}\rule{0.5\linewidth}{0.5pt}\end{center}

\begin{verbatim}
e1 e2   sequencing
e1 | e2 prioritized choice
e∗      zero or more repetitions
e+      one or more repetitions (not essential)
∼e      negation
<p>     production application
’x’     matches the character x
\end{verbatim}

\begin{center}\rule{0.5\linewidth}{0.5pt}\end{center}

But wait there's more to go over on memory! Storage and Types

storage classes - auto, static, extern, and register only one storage
class can be specified comes first in declaration

\begin{verbatim}
* storage duration - static or automatic
* scope - block or file
* linkage - external, internal, or no linkage
\end{verbatim}

auto(by default), static(), extern(says that this variable will be
defined somewhere else, likely another file)

type qualifier - const(val can't be changed abort with error if this is
attempted) and volatile(value may change outside our control)

type specifier - struct, enum, union need to be converted to regular
types(\texttt{uintN\_t} \texttt{intN\_t} \texttt{floatN\_t}) or their
pointers with typedef

\begin{center}\rule{0.5\linewidth}{0.5pt}\end{center}

\hypertarget{appendix}{%
\subsection{Appendix}\label{appendix}}

\texttt{Malloc()} basically wraps both \texttt{brk} and \texttt{mmap2}
syscalls, but it also keeps internal information about the memory that
has been allocated for use in later malloc calls or free calls. how this
is implemented falls outside the scope of a compiler: Having to move
around countless copies of the \texttt{malloc()} function in each and
every executable in the operating system is going to be slower than
Dynamically linking to \texttt{malloc()}. Executable \texttt{.o} files
aren't just a bunch of assembly code, there needs to be dynamically
linkable functions at execution time even if we do our own linking.

\hypertarget{events}{%
\subsubsection{Events}\label{events}}

Probably should write a library for events, hopefully this can be done
in a library.

The sublist method of event management, maintains a list of subscribers
for each possible event, and when that event is published it is sent out
to each of them. Currently this section is in java, but I should
translate it into C soon implemented as a C library as well, so we're
actually going to describe the operations in java, to avoid any
confusion added from also having to juggle pointers and address in
addition to the events.

thrower=publisher, throw=publish, catcher=subscriber, catch=receive

\begin{verbatim}
import java.util.*;//import of java.util.event

interface ThrowListener { public void Catch(); } //Declaration event interface type
// OR import of the interface, OR declared somewhere else in the package

/*_______________________________________________*/class Thrower {
//list of catchers & corresponding function to add/remove them in the list
    List<ThrowListener> listeners = new ArrayList<ThrowListener>();
    public void addThrowListener(ThrowListener toAdd){ listeners.add(toAdd); }
    //Set of functions that Throw Events.
        public void Throw(){ for (ThrowListener hl : listeners) hl.Catch();
            System.out.println("Something thrown");
        }
////Optional: 2 things to send events to a class that is a member of the current class
. . . go to Thrower.java file see this code . . .
}

/*_________________________________________________*/class Catcher
implements ThrowListener {//implement added to class
//Set of @Override functions that Catch Events
    @Override public void Catch() {
        System.out.println("I caught something!!");
    }
////Optional: 2 things to receive events from a class that is a member of the current class
. . . go to Catcher.java file see this code . . .
}


#                                                               #
#                                                               #
#                                                               #
\end{verbatim}

\begin{center}\rule{0.5\linewidth}{0.5pt}\end{center}

Imperative \& Declarative Programming (Programming as Planning,
Executable Specifications)(Samimi 2009)

i'm sure you have seen an imperative as well as declarative
methodologies to do programming. imperative is how implementations are
done and declarative is we've seen prolog language you know it's
meanings they're compact um they're very understandable but we need
search in order to to execute them so that they're often not practical.

part of what you're interested here to save codes as well as to add
understandability to problems is; is there a way to kind of combine both
at the same time? Which is not very common today. This is a famous quote
that i made recently that the natural combination of declaritive and
imperative is missing. A couple of methodologies that we kind of studied
in this direction:

basically if you add a a a layer of logic on top of your your your
languages to support search basically your heuristic search you can do
imperative programming with a little bit of overhead if you always
willing to add optimizations so in the in the presence of multiple
methods if there's a if there's an optimization that always tells you
based on the state which method is the is a good method to take so you
always do this check to to find out the different alternatives then you
have this little bit overhead of checking to do your normal imperative
but when you do want when you do have the you know um leverage to do
kind of search you have the option to say well explore possibilities and
find the best scenario to to do so this is what this this project is all
about you know allows you to kind of separate the optimizations and
heuristics from from the actual implementations the actions that you see
here to satisfy goals.

\begin{center}\rule{0.5\linewidth}{0.5pt}\end{center}

The need for avoiding complexity is more important in the context of
\textbf{DI} than the need for optimization.

See also:

Open, extensible composition models(Piumarta 2011b)

Association-based model of dynamic behaviour(Piumarta 2011a)

PEG-based transformer provides stages in a compiler(Piumarta 2010)

Open, Extensible Object Models(Piumarta and Warth 2008)

OMeta(Warth and Piumarta 2007)

Virtual processor: dynamic code generation(Piumarta 2004)

\href{https://www.complang.tuwien.ac.at/Diplomarbeiten/falbesoner14.pdf}{Sebastian
Falbesoner 2014 Implementing a Global Register Allocator for TCC}

how C2 works described in (Paleczny, Vick, and Click 2001)

\begin{center}\rule{0.5\linewidth}{0.5pt}\end{center}

\hypertarget{references}{%
\subsection*{References}\label{references}}
\addcontentsline{toc}{subsection}{References}

\hypertarget{refs}{}
\begin{CSLReferences}{1}{0}
\leavevmode\vadjust pre{\hypertarget{ref-10.1145ux2f800230.806984}{}}%
Chaitin, G. J. 1982. {``Register Allocation \&Amp; Spilling via Graph
Coloring.''} In \emph{Proceedings of the 1982 SIGPLAN Symposium on
Compiler Construction}, 98--105. SIGPLAN '82. New York, NY, USA:
Association for Computing Machinery.
\url{https://doi.org/10.1145/800230.806984}.

\leavevmode\vadjust pre{\hypertarget{ref-ClickNodes}{}}%
Click, Cliff. 2020. {``The Sea of Nodes and the HotSpot JIT.''}
\emph{YouTube}. \url{https://youtu.be/9epgZ-e6DUU?t=3535}.

\leavevmode\vadjust pre{\hypertarget{ref-Kay_video2009}{}}%
Kay, Alan. 2009. {``How Complex Is "Personal Computing"?''}
\emph{YouTube}. \url{https://youtu.be/HAT4iewOHDs}.

\leavevmode\vadjust pre{\hypertarget{ref-10.5555ux2f1267847.1267848}{}}%
Paleczny, Michael, Christopher Vick, and Cliff Click. 2001. {``The Java
HotspotTM Server Compiler.''} In \emph{Proceedings of the 2001 Symposium
on JavaTM Virtual Machine Research and Technology Symposium - Volume 1},
1. JVM'01. USA: USENIX Association.

\leavevmode\vadjust pre{\hypertarget{ref-10.5555ux2f1267242.1267250}{}}%
Piumarta, Ian. 2004. {``The Virtual Processor: Fast,
Architecture-Neutral Dynamic Code Generation.''} In \emph{Proceedings of
the 3rd Conference on Virtual Machine Research and Technology Symposium
- Volume 3}, 8. VM'04. USA: USENIX Association.

\leavevmode\vadjust pre{\hypertarget{ref-Piumarta_video2007}{}}%
---------. 2007. {``Building Your Own Dynamic Language.''}
\emph{YouTube}. \url{https://youtu.be/cn7kTPbW6QQ?t=2736}.

\leavevmode\vadjust pre{\hypertarget{ref-10.1145ux2f1942793.1942796}{}}%
---------. 2010. {``PEG-Based Transformer Provides Front-, Middle-and
Back-End Stages in a Simple Compiler.''} In \emph{Workshop on
Self-Sustaining Systems}, 10--20. S3 '10. New York, NY, USA: Association
for Computing Machinery. \url{https://doi.org/10.1145/1942793.1942796}.

\leavevmode\vadjust pre{\hypertarget{ref-10.1145ux2f2068776.2068779}{}}%
---------. 2011a. {``An Association-Based Model of Dynamic Behaviour.''}
In \emph{Proceedings of the 1st International Workshop on Free
Composition}. FREECO '11. New York, NY, USA: Association for Computing
Machinery. \url{https://doi.org/10.1145/2068776.2068779}.

\leavevmode\vadjust pre{\hypertarget{ref-10.1145ux2f2068776.2068778}{}}%
---------. 2011b. {``Open, Extensible Composition Models.''} In
\emph{Proceedings of the 1st International Workshop on Free
Composition}. FREECO '11. New York, NY, USA: Association for Computing
Machinery. \url{https://doi.org/10.1145/2068776.2068778}.

\leavevmode\vadjust pre{\hypertarget{ref-10.1007ux2f978-3-540-89275-5_1}{}}%
Piumarta, Ian, and Alessandro Warth. 2008. {``Open, Extensible Object
Models.''} In \emph{Self-Sustaining Systems: First Workshop, S3 2008
Potsdam, Germany, May 15-16, 2008 Revised Selected Papers}, 1--30.
Berlin, Heidelberg: Springer-Verlag.
\url{https://doi.org/10.1007/978-3-540-89275-5_1}.

\leavevmode\vadjust pre{\hypertarget{ref-Samimi_video2009}{}}%
Samimi, Hesam. 2009. {``Imperative \& Declarative Programming
(Programming as Planning, Executable Specifications).''} \emph{YouTube}.
\url{https://youtu.be/HAT4iewOHDs?t=3578}.

\leavevmode\vadjust pre{\hypertarget{ref-Warth_video2009}{}}%
Warth, Alessandro. 2009. {``Domain-Specific Languages (OMeta).''}
\emph{YouTube}. \url{https://youtu.be/HAT4iewOHDs?t=2060}.

\leavevmode\vadjust pre{\hypertarget{ref-10.1145ux2f1297081.1297086}{}}%
Warth, Alessandro, and Ian Piumarta. 2007. {``OMeta: An Object-Oriented
Language for Pattern Matching.''} In \emph{Proceedings of the 2007
Symposium on Dynamic Languages}, 11--19. DLS '07. New York, NY, USA:
Association for Computing Machinery.
\url{https://doi.org/10.1145/1297081.1297086}.

\end{CSLReferences}
