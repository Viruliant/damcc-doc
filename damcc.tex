\hypertarget{definitional-architecture-metaprogramming}{%
\section{Definitional Architecture
Metaprogramming}\label{definitional-architecture-metaprogramming}}

``The idea that you can do a programming language in 100 000 lines of
code or you might be able to do it in a page and there are advantages to
the page for sure'' --Alan Kay 2009 How Complex is ``Personal
Computing''?

\begin{center}\rule{0.5\linewidth}{0.5pt}\end{center}

\hypertarget{definitional-interpreters-di}{%
\paragraph{Definitional-Interpreters
(DI)--}\label{definitional-interpreters-di}}

convey the ideas and semantics of a language with brevity and clarity;
the advantage with defining a language with this method over a
specification document is that the result of the \textbf{DI} is
executable, so there is no question of the meaning of an expression in a
context. The need for avoiding complexity is more important in the
context of \textbf{DI} than the need for optimization.

\hypertarget{architecture}{%
\paragraph{Architecture--}\label{architecture}}

Different cpus need different executables if they use different
instruction-sets, and even if they do use the same
\textbf{ISA}(instruction-set-architecture), they may have new Extensions
that may extend an \textbf{ISA}, to let older programs run on newer
cpus.

\hypertarget{metaprogramming}{%
\paragraph{Metaprogramming--}\label{metaprogramming}}

a programming technique in which computer programs have the ability to
treat other programs as their data.

\hypertarget{c-compiler-damcc-as-the-di-of-the-c-language}{%
\paragraph{\texorpdfstring{``\texttt{C}'' Compiler (\texttt{damcc}) as
the DI of the ``\texttt{C}'' language
--}{``C'' Compiler (damcc) as the DI of the ``C'' language --}}\label{c-compiler-damcc-as-the-di-of-the-c-language}}

Rarely is a language as well put together as \texttt{C}, Mixing
\texttt{C} and assembly can often be as simple as not translating the
assembly parts of the \texttt{C} code. A lot of compromises were made to
do this; For all intents and purposes \texttt{C} is not so much a
language so much as it is basically assembly and just as dangerous.
\texttt{C} is a notoriously ``Dumb'' language, in that it will do what
exactly what \textbf{you} tell it to do; Including the set of all known
computer \emph{bugs}: memory leaks, stack overflows, disk destruction,
or worse.

\hypertarget{architecture-data-structure-c}{%
\paragraph{\texorpdfstring{Architecture Data-Structure ``\texttt{C}''
--}{Architecture Data-Structure ``C'' --}}\label{architecture-data-structure-c}}

For compiling/interpreting any language; \texttt{C} is an excellent
target architecture to use as a stepping stone to actual compiled
assembly code. ``\texttt{damcc}'' is not just a \textbf{DI} for
\texttt{C} it is also to serve as documentation for the different
\textbf{ISA}s. Users should be able to understand an architecture
quickly and easily by reading the code for the supported
\textbf{ISA}s.(as well as elf files and static/dynamic linking)
\emph{``It is better to have 100 functions operate on one data structure
than 10 functions on 10 data structures.''} --Alan Perlis

\hypertarget{stage-1-subset-of-c-as-an-intermediate-language}{%
\paragraph{Stage 1 subset of C as an intermediate
language--}\label{stage-1-subset-of-c-as-an-intermediate-language}}

Define what the \texttt{C} language is by using ``Ometa'' to convert
from preprocessed \texttt{C} to \texttt{C} or our small subset of it
that is more easily translatable to machine code. The goal here of using
this subset of \texttt{C} is to ``explicitly enumerate all the nuances
of the language'' by ``removing them''.

\begin{itemize}
\item
  The meaning of all loops translated into gotos
\item
  nested if statements translated into guard clause non-nested if
  statements
\item
  no more Arrays, Types, Struct, Union, Enum, or any other data-types
  not manipulatable with assembly; just pointer arithmetic.
\item
  even though we can replace function calls with gotos, they need to be
  left in to represent where we will leave symbol references for the
  ``\texttt{.o}'' files
\item
  copies of variables that are created by a function call if they aren't
  passed by reference, the goal here is to perhaps be Bi-directionally
  translatable.
\item
  \texttt{sizeof()} tells the number of bytes that type or array
  occupies, and \texttt{sizeof()} is known at compile time, and most of
  the time this number is fed into arithmetic functions that can also be
  calculated at compile time; which can allow optimizations here.
\item
  \texttt{\#include} lines willl be removed, files referenced by
  \texttt{\#include} will have their funtions scanned and added to a
  table of functions so that when a external function is called the
  appropriate ``\texttt{symbol()}'' call of the elf file can be made.
\item
  \textbf{Memory--} all of variables \textbf{AND} pointers Each function
  is to have all memory explicitly allocated/delocated with appropriate
  calls to:

  malloc() realloc() calloc() aligned\_alloc() free()
\end{itemize}

\hypertarget{memory-continued}{%
\paragraph{memory continued--}\label{memory-continued}}

Malloc() basically wraps both brk and mmap2 syscalls, but it also keeps
internal information about the memory that has been allocated for use in
later malloc calls or free calls. how this is implemented falls outside
the scope of a compiler: Having to move around countless copies of the
malloc() c function in each and every executable in the operating system
is going to be slower than Dynamically linking to malloc(). Executable
files aren't just a bunch of assembly code, there needs to be
dynamically linkable functions at execution time even if we do our own
linking.

\begin{center}\rule{0.5\linewidth}{0.5pt}\end{center}

\emph{``I came to realize that if I'm not writing a program, I shouldn't
use a programming language. People confuse programming with coding,
coding is to programming what typing is to writing. It's something that
involves mental effort, what you're thinking about, what you're going to
say, the words have some importance; but in some sense that even they
are secondary to the ideas. In the same way programs are built on ideas,
they have to do something and what they're supposed to do, is like what
writing is supposed to convey. If people are trying to learn programming
by being taught to code, well they're being taught writing by being
taught how to type, and that doesn't make much sense.''} --Leslie
Lamport https://youtu.be/rkZzg7Vowao

\hypertarget{ometa-metaprogramming}{%
\subsection{\texorpdfstring{\textbf{OMeta}
Metaprogramming}{OMeta Metaprogramming}}\label{ometa-metaprogramming}}

\emph{``The system that tries to find coverings of those trees pieces of
structure within the trees is itself very very simple and you can write
2 functions that do it reduce a particular piece of tree to a quantity a
register in the machine in memory location literal in the machine or
avoid just a statement that returns no value and do that by matching a
particular piece of tree against the pattern in the list of rules these
things are recursive so that structuring the trees can have sub
structure etc and what falls out at the end is a yes or no answer and
it's the inverse of the inverse of traditional parsing if you like where
a normal posit X unstructured free text and creates a structured
representation}

\begin{verbatim}
reduce(tree, startSymbol) =
    foreach rule in startSymbol.startSets
        if match(tree, rule.pattern)
            rule.action()
            return startSymbol
    return false

match(tree, pattern) =
    if (pattern.isSymbol()) return reduce(tree, pattern)
    if (tree.first != pattern.first) return false
    foreach treeElement, patternElement in tree.tail, pattern.tail
        unless match(treeElement, patternElement)
            return false
    return true
\end{verbatim}

\emph{what this bottom-up system does is it takes a structured
representation and tries to find inside quotes an optimal unstructured
equivalent which in our case are machine instructions for the for the
target machine}

\emph{One of the interesting things that we've now done is we've
described cogeneration as a tree rewriting process and what we've really
been doing throughout the whole talk is describing how various phases of
the implementation of a programming system can be described in terms of
simple tree line structures and rules rewriting rules applied to them
and we can hope in the end to create a single very small very easily
understandable system tree rewriting system in which the whole of the
language implementation problem is described.}

\emph{now if you choose to go our way and play with some of the ideas
that I'm talking about today one thing you will discover is that you
need to find a fixed point to get the recursive descriptions started and
the way that we did it was}

\emph{write a bootstrap compiler in C that could understand a little
object language we then rewrote the object language in the object
language fed it through the bootstrap compiler to create a real compiler
and then deleted the bootstrap compiler because it was now
uninteresting.}

\emph{Then everything else the the transformation rules for taking
structures applying all the functional rules to them to create
executable things within fed in and from that point on the whole system
takes off exponentially because of the simplicity the expressiveness
that is built into those 2 mutually supportive models within it.''}

-- 45:36 into presentation
``\href{https://youtu.be/cn7kTPbW6QQ}{Building Your Own Dynamic
Language}'' by Ian Piumarta
\href{https://www.piumarta.com/papers/EE380-2007-slides.pdf}{slides}
also see Alessandro Warth - 34:20 - How Complex is ``Personal
Computing''?

\begin{center}\rule{0.5\linewidth}{0.5pt}\end{center}

\hypertarget{metaprogramming-example-translator}{%
\paragraph{Metaprogramming Example
translator}\label{metaprogramming-example-translator}}

Control flow operators? maybe all we need is ``\texttt{goto}'' read this
code:

\begin{verbatim}
while (condition) statement
//the above while-loop is equivalent to the following logic
begin:
if (condition) {
    statement;
    goto begin;
}

do [loop body statement] while ([condition]);
//the above do-while-loop is equivalent to the following logic
begin:
loop body statement;
if (condition)
    goto begin;

for (variable-declaration; condition; variable-update) statement
//the above for-loop is equivalent to the following logic
variable-declaration;
begin:
    if (!condition)
        goto end;
    statement;
    variable-update;
    goto begin;
end:
    // end loop 

switch (expression) {
    case const-expr1: statement1
    case const-expr2: statement2
    case const-expr3: statement3
    default: statements
}
//the above switch is equivalent to the following if-else logic
type value = expression
if (value == const-expr1)
    statement1
else if (value == const-expr2)
    statement2
else if (value == const-expr3)
    statement3
else
    statement0

#                                                               #
#                                                               #
#                                                               #
#                                                               #
#                                                               #
#                                                               #
#                                                               #
#                                                               #
#                                                               #
#                                                               #
#                                                               #
#                                                               #
#                                                               #
#                                                               #
#                                                               #
#                                                               #
#                                                               #
#                                                               #
#                                                               #
#                                                               #
#                                                               #
#                                                               #
//something else
if (n > 0) {
    if (a > b)
        z = a;
}
else
    z = b;
\end{verbatim}

\hypertarget{elf-format-relocatable-object-.o-files}{%
\subsection{\texorpdfstring{(ELF Format) Relocatable Object
``\texttt{.o}'' Files
--}{(ELF Format) Relocatable Object ``.o'' Files --}}\label{elf-format-relocatable-object-.o-files}}

Relocatable object files are files that are in the ``elf'' executable
and linkable format. To Dynamically link to each foreign function call
requires parsing header files to see which header provides each function
that is used in the current file, to properly call the function and
throw an error if a function isn't available. ``\texttt{.o}'' files are
binary data assembly with symbols packed in, based on a specification.
This is the format for linux executables and libraries, These files are
the raw inputs to ``linkers''. Translating from ``\texttt{C}'' to
``\texttt{.o}'' linkable binary files is the goal here and the
definition of compiling.

\hypertarget{stage1-notes}{%
\subsubsection{Stage1 notes}\label{stage1-notes}}

\hypertarget{clean-and-convoluted}{%
\paragraph{\texorpdfstring{\emph{``Clean''} and
convoluted--}{``Clean'' and convoluted--}}\label{clean-and-convoluted}}

Recall from the introduction after Stage1 is complete there will be no
need to:

\begin{itemize}
\tightlist
\item
  parse the \texttt{sizeof()} command
\item
  parse Arrays, Types, Struct, Union, Enum
\item
  parse control flow statements switches loops etc.
\item
  create any variables -- the c code will be requesting memory for it's
  own variables and pointers the appropriate function calls
  (\texttt{malloc()} etc.).
\end{itemize}

\hypertarget{stage2---on-the-shoulders-of-giants}{%
\subsubsection{Stage2 - on the shoulders of
giants}\label{stage2---on-the-shoulders-of-giants}}

58:57 - 60:54 into \href{https://youtu.be/9epgZ-e6DUU}{Cliff Click -
HotSpot ``C2'' JIT} ``The Java Hotspot Server Compiler'' high level
overview of how C2 works Click, Paleczny, Vick

\emph{``\textbf{Outline how The architecture description files, in how
in general made the compiler suitable for many platforms all at once:}
So there are some things there that I wouldn't do again including the
burrs technology there is an architecture file that is used to describe
a CPU architecture including the set of the kind of instructions and a
mapping from the sea to ideal nodes to machine equivalents and the
registers that are allowed and the encoding for the machine code
encodings for them is sort of the common set of things there's a bunch
of the things that are in there that are all these fine-grained details
but the big pieces are:}

\emph{here's a mapping from idealized nodes which is what the eye are
mostly runs on to get under code gen to hardware instructions and the
registers that are allowed both on each individual hardware instruction
input and overall what are those set of arc registers are allowed in the
system and there is a tool which reads this file which is enough funny
in its own little format and spits out C code that is in compiled in
with the rest of the system bills what's called a burrs pattern matching
which does a optimal for some definition of optimal mapping of Hardware
instructions to ideal nodes by doing like an overlapping tree like like
usually a hardware node covers two to five ideal nodes and so there's
some tiling operation that covers them all including if it eight this
region then those edges in between don't have to register allocated but
all the edges on the outside do and then what all the registers that are
involved and the registers turn into basically a bit set for every input
and output of allowed registers and the register allocator Grox that bit
set and uses it to pick the right registers so that gets into registrar
in pretty heavily right away because that's how you describe inputs to
the register allocator''}

\hypertarget{inline-arithmetic-functions-must-be-written-for-each-architecture}{%
\paragraph{Inline Arithmetic functions must be written for each
architecture
--}\label{inline-arithmetic-functions-must-be-written-for-each-architecture}}

Arithmetic operations like ``\texttt{+}'' ``\texttt{-}'' ``\texttt{*}''
``\texttt{/}'' ``\texttt{mod}'' ``\texttt{expt}'' etc. will be
translated to a equivalent call e.g.~``\texttt{plus}'' that can be
``inlined'' FOR EACH TYPE. Traditionally the instructions of a function
have there own place in memory and a jump to that section is performed
followed by a jump back to wherever the function was called from. Inline
functions mean the whole code of the function is written verbatim each
time it is called without any jump instructions. This is the reason
programmers have the ``\texttt{inline}'' primitive, the c code the user
feeds to the compiler and if the linker decides to honor the
``\texttt{inline}'' primitive is where the decision is to be made.

\begin{verbatim}
// Static Inline function for each possible compare or jump
// for the current architecture in assembly
static inline int foo(){
    return 2;
}
\end{verbatim}

At each stage of compilation/translation \texttt{damcc} data will remain
in C until there is nothing left but assembly.

``address of'' operator: ``\texttt{\&}'' - returns address in memory of
a variable ``dereference'' operator: ``\texttt{*}'' - Grab data of type
of the argument at the given memory location ``ffi''-foreign function
interface ``efi''-external function interface

\begin{center}\rule{0.5\linewidth}{0.5pt}\end{center}

\hypertarget{executable-linkers-are-basically-just-home-theater-setups-}{%
\paragraph{Executable Linkers are basically just home theater
setups-}\label{executable-linkers-are-basically-just-home-theater-setups-}}

Executable Linkers are basically just home theater setups CS361 Chris
Kanich
https://youtu.be/eQ0KOT\_J8Sk?list=PLhy9gU5W1fvUND\_5mdpbNVHC1WCIaABbP

in certain situations with low memory budgets other techniques might be
used --https://github.com/avrdudes/avr-libc

\hypertarget{external-function-interface}{%
\paragraph{external function
interface}\label{external-function-interface}}

To allow assembly functions in our compiled code to to be called from
another program, we have tor create an external function interface.

https://stackoverflow.com/questions/13901261/calling-assembly-function-from-c

\begin{center}\rule{0.5\linewidth}{0.5pt}\end{center}

\hypertarget{appendix}{%
\subsection{Appendix}\label{appendix}}

Sebastian Falbesoner 2014 Implementing a Global Register Allocator for
TCC

https://www.complang.tuwien.ac.at/Diplomarbeiten/falbesoner14.pdf

\hypertarget{events}{%
\subsection{Events}\label{events}}

Probably should write a library for events, hopefully this can be done
in a library.

The sublist method of event management, maintains a list of subscribers
for each possible event, and when that event is published it is sent out
to each of them. Currently this section is in java, but I should
translate it into C soon implemented as a C library as well, so we're
actually going to describe the operations in java, to avoid any
confusion added from also having to juggle pointers and address in
addition to the events.

\begin{verbatim}
import java.util.*;//import of java.util.event

interface ThrowListener { public void Catch(); } //Declaration event interface type
// OR import of the interface, OR declared somewhere else in the package

/*_____________________________________________________________*/class Thrower {
//list of catchers & corresponding function to add/remove them in the list
    List<ThrowListener> listeners = new ArrayList<ThrowListener>();
    public void addThrowListener(ThrowListener toAdd){ listeners.add(toAdd); }
    //Set of functions that Throw Events.
        public void Throw(){ for (ThrowListener hl : listeners) hl.Catch();
            System.out.println("Something thrown");
        }
////Optional: 2 things to send events to a class that is a member of the current class
. . . go to Thrower.java file see this code . . .
}

/*_______________________________________________________________*/class Catcher
implements ThrowListener {//implement added to class
//Set of @Override functions that Catch Events
    @Override public void Catch() {
        System.out.println("I caught something!!");
    }
////Optional: 2 things to receive events from a class that is a member of the current class
. . . go to Catcher.java file see this code . . .
}
\end{verbatim}
